\section{Personnel, Resources, Risks, and Timeline}

I propose to carry out this research over a five year period with a graduate student research as-
sistant and half-time postdoctoral researcher at UC San Diego. Our team will leverage existing
collaborations (as expressed in the letters of collaboration in App. 7) with Fermilab (J. Berryhill,
L. Gray, S. Jindariani, and N. Tran), MIT (P. Harris), CERN (M. Pierini), and industry partners
including Xilinx (E. Delaye) and Microsoft (T. Way).
My background and technical expertise are well matched to the goals of the proposed research
program. As a developer of GNNs for jet classification [4], a creator of tools to deploy low-latency
neural networks for particle physics [2], and a contributor to and section editor of the technical
design report for the CMS L1 trigger upgrade [22], I am singularly qualified to expand the fast
AI program to new network architectures and difficult real-time reconstruction tasks. At the time
of writing, I have also been nominated for the role of a co-convener of the new CMS Machine
Learning group, tasked with coordinating, reviewing, and assisting with ML efforts in all stages
of the experiment.
To accomplish the development and hardware implementation of these algorithms, I will
leverage the resources and collective expertise of UC San Diego’s Physics, Electrical and Com-
puter Engineering, and Computer Science and Engineering departments as well as the Califor-
nia Institute for Telecommunications and Information Technology (Calit2), Pacific Research Plat-
form (PRP), San Diego Supercomputer Center (SDSC), and the Halıcıo  ̆glu Data Science Institute
(HDSI). These resources include GPUs and other specialized AI hardware to quickly train GNNs
and other algorithms for particle physics applications. PRP also provides access to Xilinx software
licenses and virtual machines for compiling firmware. I will also expand collaborations with AI
researchers in HDSI (H. Su) focusing on 3D deep learning, physics-aware algorithms, and neural
network compression.
This project is not without some risk. It is possible that AI developments will not bring mean-
ingful improvements to the H → WW analysis or trigger reconstruction, though I expect it will
based on previous experience with improvements from AI algorithms for H → bb identification
and other tasks. For the trigger, it is also possible that the developed AI algorithms will be too
resource intensive or not meet the latency constraints. To mitigate this risk, I plan to extensively
study compression techniques [42–46], as well as alternative network architectures designed to
be more efficient [47–49]. Moreover, we have already demonstrated that modified particle-flow
and PUPPI algorithms can run on FPGAs in less than 750 ns as necessitated by the L1 trigger [21,
22]. For this reason, I am confident that complex algorithms can meet the L1 trigger requirements.
Nonetheless, understanding the feasibility of these AI algorithms for real-time applications is a
meaningful deliverable of the project.
A year-by-year timeline with the major proposed tasks for each personnel member and annual
deliverables follows. Schematically, the timeline is shown in Fig. 9. Delays in the LHC and HL-
LHC upgrade schedule may impact our research plan. However, for the H → WW analyses, we
can leverage the existing Run 2 CMS data and simulation to accomplish two out of three of the
major goals (the remaining goal requires Run 3 data). Similarly, for the trigger component, two
out of three of our principal goals can be accomplished independently of the HL-LHC upgrade
schedule. The last goal relies on the availability of pre-production HL-LHC trigger processing
boards, expected to be ready for acquisition in 2021. If these boards are not available in time to
perform our tests, we can instead acquire and use a prototype version of the board that is available
now. Additional time (“float”) is included in the project schedule to account for these potential
delays and other contingencies.
Year 1 (September 2020 - August 2021): In the first year, the group will focus on developing and
commissioning ML models for boosted H → WW identification and setting up the physical lab
space for L1 trigger development. The timeline allows for six months to recruit the postdoctoral
research associate and assumes the new graduate research assistant will begin research full time
in the spring quarter of 2021. The postdoctoral research associate will focus their efforts on the L1
trigger tasks, while the graduate student will initially focus more on data analysis and the boosted
H →WW measurements. However, both will collaborate and remain involved in all topics.
• PI (Javier Duarte): Coordination of AI, analysis, and trigger efforts; Leadership as a co-
convener within ML or L1 trigger groups; Recruitment of a postdoctoral research associate;
Acquisition of FPGA evaluation boards, L1 trigger processing board, and an AdvancedTCA
shelf for real-time AI trigger test stand at UC San Diego; Set up of ML training framework
and guidance in ML studies for H →WW reconstruction and identification in CMS.
• Postdoctoral research associate (TBN): Initial set up and testing of real-time AI trigger test
stand; Supervision of AI algorithms for H →WW reconstruction and identification.
• Graduate research assistant (TBN): Development, training, and commissioning of GNN
models for high-pT H →WW identification in CMS.

Deliverable #1: Commissioned AI algorithm for boosted H → WW reconstruction and
identification in CMS.
Year 2 (September 2021 - August 2022): In the second year, the group will focus on analysis of the
H → WW pT spectrum, leading to a timely publication with Run 2 data. In addition, Run 3 will
begin with a partial data-taking year in 2021 and a full year in 2022, allowing us to start prelim-
inary investigations with Run 3 data. The group will also prepare the simulated data samples to
train AI algorithms for particle reconstruction and pileup mitigation. With these samples, pro-
totype GNN algorithms for application in the trigger will be developed, trained, evaluated, and
presented to the CMS collaboration.
• PI (Javier Duarte): Coordination of AI, analysis, and trigger efforts; Leadership as a co-
convener within ML or L1 trigger groups; Guidance in background estimation and statistical
analysis techniques for Run 2 H → WW analysis; Conceptualizaton of initial ML models;
Guidance in ML architecture search and optimization.
• Postdoctoral research associate (TBN): Supervision of analysis efforts; Preparation of train-
ing data samples for L1 tasks based on Run 4 simulation; Development and training of pro-
totype GNN models for L1 particle reconstruction and pileup mitigation based on Run 4
simulation.
• Graduate research assistant (TBN): Analysis and measurement of boosted H → WW pT
spectrum with full Run 2 data; Combination with measurements of pT spectrum in H →bb,
H →ZZ, and H →γγ final states; Preliminary studies with first Run 3 data.
• Deliverable #2: CMS Run 2 boosted H →WW pT measurement and combination.
Year 3 (September 2022 - August 2023): In the third year, our group’s primary focus will shift to
development of AI algorithms for the HL-LHC L1 trigger. With greater availability of Run 3 data
Page 13 of 37Proposal Tracking Number: 253412       Award Number: N/APage Number: 31
IV PERSONNEL, RESOURCES, RISKS, AND TIMELINE
and simulation, we will also retrain of AI algorithms for boosted H → WW reconstruction and
identification, and investigate analysis techniques for the pT measurement.
• PI (Javier Duarte): Coordination of AI, analysis, and trigger efforts; Guidance in ML archi-
tecture search and optimization.
• Postdoctoral research associate (TBN): Continued development, refinement, and compres-
sion of GNN models for particle reconstruction and pileup mitigation based on Run 4 simu-
lation.
• Graduate research assistant (TBN): Retraining of boosted H → WW identification and re-
construction algorithms with Run 3 data and simulation; Investigation of compression tech-
niques for FPGA implementations of AI algorithms.
• Deliverable #3: Preliminary AI algorithms for HL-LHC L1 trigger applications focusing on
PF reconstruction and pileup mitigation.
Year 4 (September 2023 - August 2024): In the fourth year, our group’s effort will be shared between
continued development of AI algorithms for the HL-LHC L1 trigger and preliminary analysis and
measurements of boosted H →WW with a partial Run 3 data set.
• PI (Javier Duarte): Coordination of AI, analysis, and trigger efforts; Guidance in develop-
ment of HLS for GNN models on FPGAs; Set up and support of algorithm tests on HL-LHC
trigger boards at UC San Diego.
• Postdoctoral research associate (TBN): Integration of novel AI models, including GNNs, into
hls4ml for conversion to firmware; Preliminary testing of fast AI algorithms in FPGAs in
local test stand at UC San Diego for HL-LHC trigger applications.
• Graduate research assistant (TBN): Analysis and measurement of boosted H → WW pT
spectrum with a partial Run 3 data set.
• Deliverable #4: Expanded tools (hls4ml) for generation of real-time AI FPGA firmware.
Year 5 (September 2024 - August 2025): In the fifth year, our group’s effort will again be shared
between L1 trigger studies and Run 3 data analysis. For the L1 trigger, we will focus on a realistic
hardware demonstration of the algorithms using the test stand established at UC San Diego. We
will also contribute to the combined Run 2 and Run 3 measurement of the boosted H pT spectrum
by leading the H → WW channel, and combining with the H → bb, H → γγ, and H → ZZ
channels.
• PI (Javier Duarte): Coordination of AI, analysis, and trigger efforts; Support of algorithm
tests on HL-LHC trigger boards at UC San Diego; Guidance in background estimation and
statistical analysis techniques for Run 3 H →WW analysis.
• Graduate research assistant (TBN): Analysis and measurement of boosted H → WW pT
spectrum with full Run 3 combined data sets; Combination with Run 2 data and other decay
channels H → bb, H → ZZ, and H → γγ; Participation in AI algorithm FPGA tests at UC
San Diego; Preparations for installation and commissioning of L1 trigger hardware at CERN.
• Postdoctoral research associate (TBN): Demonstration of fast GNN algorithms in FPGAs in
local test stands at UC San Diego for HL-LHC L1 trigger applications; Integration of AI algo-
rithms for particle-flow and pileup mitigation into baseline HL-LHC L1 menu; Preparations
for installation and commissioning of L1 trigger hardware at CERN.
Page 14 of 37Proposal Tracking Number: 253412       Award Number: N/APage Number: 32
V SUMMARY AND OUTLOOK
• Deliverable #5: Demonstration of real-time AI in UC San Diego L1 trigger test stands
• Deliverable #6: CMS combined Run 2 and Run 3 boosted H pT measurement
At the conclusion of the five-year project period, our research group will be well-positioned to
lead further analysis efforts including searches for boosted HH →bbWW. In addition, we will be
ready to take a leading role in the installation and commissioning of the upgraded L1 trigger at
CERN scheduled to take place in 2026 during Long Shutdown 3 (LS3).